%package list
\documentclass{article}
\usepackage[top=3cm, bottom=3cm, outer=3cm, inner=3cm]{geometry}
\usepackage{multicol}
\usepackage{graphicx}
\usepackage{url}
%\usepackage{cite}
\usepackage{hyperref}
\usepackage{array}
%\usepackage{multicol}
\newcolumntype{x}[1]{>{\centering\arraybackslash\hspace{0pt}}p{#1}}
\usepackage{natbib}
\usepackage{pdfpages}
\usepackage{multirow}
\usepackage[normalem]{ulem}
\useunder{\uline}{\ul}{}
\usepackage{svg}
\usepackage{xcolor}
\usepackage{listings}
\lstdefinestyle{ascii-tree}{
    literate={├}{|}1 {─}{--}1 {└}{+}1 
  }
\lstset{basicstyle=\ttfamily,
  showstringspaces=false,
  commentstyle=\color{red},
  keywordstyle=\color{blue}
}
%\usepackage{booktabs}
\usepackage{caption}
\usepackage{subcaption}
\usepackage{float}
\usepackage{array}

\newcolumntype{M}[1]{>{\centering\arraybackslash}m{#1}}
\newcolumntype{N}{@{}m{0pt}@{}}


%%%%%%%%%%%%%%%%%%%%%%%%%%%%%%%%%%%%%%%%%%%%%%%%%%%%%%%%%%%%%%%%%%%%%%%%%%%%
%%%%%%%%%%%%%%%%%%%%%%%%%%%%%%%%%%%%%%%%%%%%%%%%%%%%%%%%%%%%%%%%%%%%%%%%%%%%
\newcommand{\itemEmail}{dchoquecondoas@unsa.edu.pe}
\newcommand{\itemStudent}{Daniela Choquecondo A.}
\newcommand{\itemCourse}{Pweb2}
\newcommand{\itemCourseCode}{20223023}
\newcommand{\itemSemester}{III}
\newcommand{\itemUniversity}{Universidad Nacional de San Agustín de Arequipa}
\newcommand{\itemFaculty}{Facultad de Ingeniería de Producción y Servicios}
\newcommand{\itemDepartment}{Departamento Académico de Ingeniería de Sistemas e Informática}
\newcommand{\itemSchool}{Escuela Profesional de Ingeniería de Sistemas}
\newcommand{\itemAcademic}{2023 - A}
\newcommand{\itemInput}{Del 2 Junio 2023}
\newcommand{\itemOutput}{Al 5 Junio 2023}
\newcommand{\itemPracticeNumber}{03}
\newcommand{\itemTheme}{Python}
%%%%%%%%%%%%%%%%%%%%%%%%%%%%%%%%%%%%%%%%%%%%%%%%%%%%%%%%%%%%%%%%%%%%%%%%%%%%
%%%%%%%%%%%%%%%%%%%%%%%%%%%%%%%%%%%%%%%%%%%%%%%%%%%%%%%%%%%%%%%%%%%%%%%%%%%%

\usepackage[english,spanish]{babel}
\usepackage[utf8]{inputenc}
\AtBeginDocument{\selectlanguage{spanish}}
\renewcommand{\figurename}{Figura}
\renewcommand{\refname}{Referencias}
\renewcommand{\tablename}{Tabla} %esto no funciona cuando se usa babel
\AtBeginDocument{%
	\renewcommand\tablename{Tabla}
}

\usepackage{fancyhdr}
\pagestyle{fancy}
\fancyhf{}
\setlength{\headheight}{30pt}
\renewcommand{\headrulewidth}{1pt}
\renewcommand{\footrulewidth}{1pt}
\fancyhead[L]{\raisebox{-0.2\height}{\includegraphics[width=3cm]{img/logo_episunsa.png}}}
\fancyhead[C]{\fontsize{7}{7}\selectfont	\itemUniversity \\ \itemFaculty \\ \itemDepartment \\ \itemSchool \\ \textbf{\itemCourse}}
\fancyhead[R]{\raisebox{-0.2\height}{\includegraphics[width=1.2cm]{img/logo_abet}}}
\fancyfoot[L]{Estudiante Juan Perez Perez}
\fancyfoot[C]{\itemCourse}
\fancyfoot[R]{Página \thepage}

% para el codigo fuente
\usepackage{listings}
\usepackage{color, colortbl}
\definecolor{dkgreen}{rgb}{0,0.6,0}
\definecolor{gray}{rgb}{0.5,0.5,0.5}
\definecolor{mauve}{rgb}{0.58,0,0.82}
\definecolor{codebackground}{rgb}{0.95, 0.95, 0.92}
\definecolor{tablebackground}{rgb}{0.8, 0, 0}

\lstset{frame=tb,
	language=bash,
	aboveskip=3mm,
	belowskip=3mm,
	showstringspaces=false,
	columns=flexible,
	basicstyle={\small\ttfamily},
	numbers=none,
	numberstyle=\tiny\color{gray},
	keywordstyle=\color{blue},
	commentstyle=\color{dkgreen},
	stringstyle=\color{mauve},
	breaklines=true,
	breakatwhitespace=true,
	tabsize=3,
	backgroundcolor= \color{codebackground},
}

\begin{document}
	
	\vspace*{10px}
	
	\begin{center}	
		\fontsize{17}{17} \textbf{ Informe de Laboratorio \itemPracticeNumber}
	\end{center}
	\centerline{\textbf{\Large Tema: \itemTheme}}
	%\vspace*{0.5cm}	

	\begin{flushright}
		\begin{tabular}{|M{2.5cm}|N|}
			\hline 
			\rowcolor{tablebackground}
			\color{white} \textbf{Nota}  \\
			\hline 
			     \\[30pt]
			\hline 			
		\end{tabular}
	\end{flushright}	

	\begin{table}[H]
		\begin{tabular}{|x{4.7cm}|x{4.8cm}|x{4.8cm}|}
			\hline 
			\rowcolor{tablebackground}
			\color{white} \textbf{Estudiante} & \color{white}\textbf{Escuela}  & \color{white}\textbf{Asignatura}   \\
			\hline 
			{\itemStudent \par \itemEmail} & \itemSchool & {\itemCourse \par Semestre: \itemSemester \par Código: \itemCourseCode}     \\
			\hline 			
		\end{tabular}
	\end{table}		
	
	\begin{table}[H]
		\begin{tabular}{|x{4.7cm}|x{4.8cm}|x{4.8cm}|}
			\hline 
			\rowcolor{tablebackground}
			\color{white}\textbf{Laboratorio} & \color{white}\textbf{Tema}  & \color{white}\textbf{Duración}   \\
			\hline 
			\itemPracticeNumber & \itemTheme & 04 horas   \\
			\hline 
		\end{tabular}
	\end{table}
	
	\begin{table}[H]
		\begin{tabular}{|x{4.7cm}|x{4.8cm}|x{4.8cm}|}
			\hline 
			\rowcolor{tablebackground}
			\color{white}\textbf{Semestre académico} & \color{white}\textbf{Fecha de inicio}  & \color{white}\textbf{Fecha de entrega}   \\
			\hline 
			\itemAcademic & \itemInput &  \itemOutput  \\
			\hline 
		\end{tabular}
	\end{table}
	
	\section{Tarea}
	\begin{itemize}		
 \item \item Implemente los métodos de la clase Picture. Se recomienda que implemente la clase picture por etapas, probando realizar los dibujos que se muestran en la siguiente preguntas.
		\item Usando únicamente los métodos de los objetos de la clase Picture, dibuje las figuras (invocando el método "draw()").
	\end{itemize}


	\section{Equipos, materiales y temas utilizados}
	\begin{itemize}
		\item Sistema operativo Windows 11.
		\item Librería PyGame.
		\item Python 3.10.11.
		\item Git 2.39.2.
		\item Cuenta en GitHub con el correo institucional (GabSoto).
		\item TextMaker / GhostScript / MikTex
		\item Programación Orientada a Objetos.
		\item Manejo de matrices unidimensionales.
	\end{itemize}


 \section{URL de Repositorio Github}
	\begin{itemize}
		\item URL del Repositorio GitHub para clonar o recuperar.
		\item \url{https://github.com/Dan-Ar5/Pweb2-Lab4.git}
	\end{itemize}
		
	\section{Resolución}
	\begin{itemize}
		\item Ejercicio a: 
	\begin{figure}[H]
		\centering
		\includegraphics[width=0.8\textwidth,keepaspectratio]{img/ejercicio2.png}
		%\includesvg{img/automata.svg}
		%\label{img:mot2}
		%\caption{Product backlog.}
	\end{figure}
	\end{itemize}
 \begin{itemize}
		\item Ejercicio b:
	\begin{figure}[H]
		\centering
		\includegraphics[width=0.8\textwidth,keepaspectratio]{img/ejercicio2b.png}
		%\includesvg{img/automata.svg}
		%\label{img:mot2}
		%\caption{Product backlog.}
	\end{figure}
	\end{itemize}
 \begin{itemize}
		\item Ejercicio c:
	\begin{figure}[H]
		\centering
		\includegraphics[width=0.8\textwidth,keepaspectratio]{img/ejercicio2c.png}
		%\includesvg{img/automata.svg}
		%\label{img:mot2}
		%\caption{Product backlog.}
	\end{figure}
	\end{itemize}
\begin{itemize}
		\item Ejercicio d:
	\begin{figure}[H]
		\centering
		\includegraphics[width=0.8\textwidth,keepaspectratio]{img/ejercicio2d.png}
		%\includesvg{img/automata.svg}
		%\label{img:mot2}
		%\caption{Product backlog.}
	\end{figure}
	\end{itemize}
	\begin{itemize}
		\item Ejercicio e:
	\begin{figure}[H]
		\centering
		\includegraphics[width=0.8\textwidth,keepaspectratio]{img/ejercicio2e.png}
		%\includesvg{img/automata.svg}
		%\label{img:mot2}
		%\caption{Product backlog.}
	\end{figure}
	\end{itemize}
 \begin{itemize}
		\item Ejercicio f:
	\begin{figure}[H]
		\centering
		\includegraphics[width=0.8\textwidth,keepaspectratio]{img/ejercicio2f.png}
		%\includesvg{img/automata.svg}
		%\label{img:mot2}
		%\caption{Product backlog.}
	\end{figure}
	\end{itemize}
  \begin{itemize}
		\item Ejercicio g:
	\begin{figure}[H]
		\centering
		\includegraphics[width=0.8\textwidth,keepaspectratio]{img/ejercicio2g.png}
		%\includesvg{img/automata.svg}
		%\label{img:mot2}
		%\caption{Product backlog.}
	\end{figure}
	\end{itemize}
 \item 
	\section{Cuestionario}
	\begin{itemize}
 \item ¿Qué son los archivos *.pyc? 
  \item Los archivos ".pyc" son archivos generados en Python al compilar el código fuente, con el propósito de optimizar el rendimiento durante la ejecución. Estos archivos contienen una representación de bajo nivel llamada bytecode, que permite al intérprete cargar directamente el código compilado en lugar de analizar el código fuente en cada ejecución. Esto resulta en una ejecución más rápida del programa y una reducción del tiempo de carga en ejecuciones posteriores.
  
\item ¿Para qué sirve el directorio pycache?
  \item El directorio "pycache" en Python almacena los archivos .pyc, generados durante la compilación de bytecode, para mejorar el rendimiento. Actúa como una carpeta de caché que permite un acceso más rápido a los archivos de bytecode en ejecuciones posteriores. Se crea automáticamente en el mismo lugar que los archivos .py correspondientes.

  \item ¿Cuáles son los usos y lo que representa el subguión en Python?
  \item El subguión en Python tiene varios usos y representaciones:
  \item Como variable especial: En el intérprete interactivo, representa el resultado de la última expresión evaluada.
  \item Como convención de nomenclatura: Se utiliza para indicar que un método o atributo es de uso interno y no debe ser accedido directamente desde fuera de la clase.
  \item Como nombre de archivo: Al usar un subguión como nombre de archivo, se indica que el módulo o paquete es considerado como privado y no se debería importar directamente desde otros lugares.

\url{https://github.com/Dan-Ar5/Pweb2-Lab4.git}
	\end{itemize}
 
 \item 
	\section{URL de Repositorio Github}
	\begin{itemize}
 \item Repositorio del estudiante: Daniela Mabel Choquecondo Aspilcueta:    
\url{https://github.com/Dan-Ar5/Pweb2-Lab4.git}
	\end{itemize}

\section{Referencias}
\begin{itemize}			
	\item \url{https://www.w3schools.com/python/python_reference.asp}
	\item \url{https://docs.python.org/3/tutorial}
\end{itemize}	
	
%\clearpage
%\bibliographystyle{apalike}
%\bibliographystyle{IEEEtranN}
%\bibliography{bibliography}
			
\end{document}